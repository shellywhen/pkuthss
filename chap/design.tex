\chapter{双分辨率词云的设计目标}

为了合理提出双分辨率词云及其自定义创制系统的视觉设计,本章节将讨论双分辨率词云所关注的任务与使用情境,结合可视化中相关的重要设计准则,推导出设计要求,以此在众多首要的设计选择中明确具体的方案,明确设计的方向。

\section{任务}
双分辨率词云是一种通用的文本可视化方法,其核心是延拓现有词云,提供对上下文的支持,在静态图像中额外编码关键词对应的语境。我们认为,无论是出于探索式数据分析的需求还是出于可视化具有关联的文本的需要,双分辨率词云都能发挥一定的作用。此外,无论是词云还是多分辨率图像,他们都能够娱乐观众,兼具艺术性和趣味性。而脱胎于此的双分辨率词云亦应继承这一良好特性。

\begin{enumerate}[leftmargin=*]
	\renewcommand{\labelenumi}{\textbf{T\theenumi.}}
	\renewcommand{\labelenumii}{\textbf{ T\theenumi.\arabic{enumii}}}

	\item \textbf{探索式数据分析}---尽管词云不涉及复杂的自然语言处理技术,仅凭借词频粗略地概括文本内容,但它很适用探索式数据分析的情境。当分析者对数据一无所知,尚未建立任何猜想或假设,关键词这种回归原始数据的底层信息能够使分析者快速构建对数据自身的理解,了解文本中的主要话题。而双分辨率词云则是进一步提供了具体的语境,使得分析者可通过递进的方式进行探索,更为准确地理解数据。
%	他们所关心的问题可能包括
%	\begin{itemize}
%		\item 这一文本有什么内容?
%	\end{itemize}

   \item \textbf{层次化文本数据展示}---双分辨率词云的形式除了能够展示关键词及其上下文,也可用于展示其他具有二重层次关联的文本数据。例如外层词云对应主题,内层词云对应主题下的关键词的情况。它能将抽象的数据处理结果可视化,辅助交流与沟通。

%   在这一任务的驱动下,双分辨率词云应解决
%   \begin{itemize}
%   	\item
%   \end{itemize}
%
   \item \textbf{寓教于乐}---所谓“一图胜千言”,生动的信息图表(Inforgraphics)往往能激发观众的兴趣,带来别样的浏览体验~\supercite{Munzner2014}。我们希望双分辨率词云具有寓教于乐的功能,在反映数据的同时兼具艺术性与趣味性,让随机浏览的人们沉浸其间,使创作者能更好地向大众传达词云所隐含的信息。

\end{enumerate}


\section{设计准则}

除了以上对分析情境及对应任务的考量,我们在双分辨率词云的设计中也需要考虑一些普适的设计准则与认知规律。
\begin{enumerate}[leftmargin=*]
	\renewcommand{\labelenumi}{\textbf{P\theenumi.}}
	\renewcommand{\labelenumii}{\textbf{ P\theenumi.\arabic{enumii}}}
	\item \textbf{可视化设计标准}
		\begin{enumerate}
		\item \textbf{Gestalt理论}---Gestalt理论~\supercite{Wagemans2012ACO}描述了人们在感知或理解图像时,根据某些特征将元素聚集所表现出来的规律。通过这种机制,人们得以简化图像的模式、理解其内涵。其中,与本研究相关的规律有以下三点。
		\begin{itemize}
			\item \textit{相似性聚集}:相似的元素会被视为一类。我们习惯于会不自觉将形状或色彩等基本视觉属性接近的元素划归为一类,认为他们具有相同的作用。
			\item \textit{相邻性聚集}:物理位置上靠近的元素会被视为一类。而对于彼此存在巨大空隙的元素,就算他们在外观上具有一定相似性,我们也倾向于认为他们之间无关。即是说,相邻性聚集的作用远远大于相似性聚集。
			\item \textit{共享空间聚集}:处于同一封闭区域内部的元素会被认为同属一类。
		\end{itemize}
		\item \textbf{自顶而下的探索流程}---可视化领域广泛认可``先概览,通过聚焦和过滤进行筛选,最后提供细节信息''~\supercite{Shneiderman96}的交互探索流程,大量应用均遵循着这一原则。概览让用户首先对数据产生基本的认识,定位到感兴趣的数据子集,再从细节中完成更为精细的分析任务。
		\item \textbf{使用更少的视图。}在具有多个视图的可视化中,应尽可能减少所使用的视图数目~\supercite{Baldonado2000}。尽管利用交互切换视图能帮助用户在数据概览中渐进地定位到关键区域,对更多的细节展开探索,但过多的视图切换伴随着额外的认知开销,会产生学习与交互成本,使用户难以构建对数据的宏观理解(Mental Map),最终影响数据分析的效率。
	\end{enumerate}
	\item \textbf{美观标准}
			\begin{enumerate}
	\item \textbf{空间覆盖率高}---参考Wang等人~\supercite{Wang2020}对词云美观性的评价指标,我们以词云中空白区域像素个数占所有像素个数的比例作为度量。
		\item \textbf{布局均匀}---在词云的布局中,无关联的文本应均匀地布局在图片上。换而言之,词与词之间的空隙是均匀分布的。
		\item \textbf{字符清晰可辨}---两层词云各不影响。在最理想的情形下,对于特定的距离范围,只有父词云或子词云可见,且同级词云均可清晰辨别。
	\end{enumerate}
\end{enumerate}

\section{设计选择}
\paragraph{大屏还是小屏,静态还是交互?}
双分辨率词云的初衷是为基本的词云提供上下文信息,使人们能够结合语境更好地理解数据。以此为出发点,我们其实还有许多设计选择。由于除了客观的人眼感知能力与计算机硬件性能,可视化工具展示数据的能力(即视觉可扩展性\supercite{Eick2002}) 还受到屏幕分辨率、可视系统的交互性、可视化隐喻内涵等因素的影响,我们由此来考虑备选设计。

首先是交互性。在可视分析中,许多系统为用户提供了多种相互关联的视图,让他们通过自行筛选来浏览更多的细节,逐渐发掘数据背后的知识。类似地,我们可以提出这样的一个设计:只显示父词云,再通过动态查询~\supercite{DBLP:journals/software/Schneiderman94}展示相应的子词云信息,如返回一个列表。结合我们的任务,答案是否定的。这种渐进式的探索需要用户确定感兴趣的区域,再去执行筛选而改变视图,缺乏即时性。对于向多人展示的场景(T2),无法体现数据的全貌,缺乏说服力。而对受众随机浏览的场景(T3),由于交互不是显式存在的,极有可能会被忽略,最终与静态普通词云的效果无异,背离了初衷。

而由于文本的非结构化特性,为其换用图形表征是较为困难的,因此不予考虑。最终,我们选择在大屏上静态地同时显示上下层词云。提高屏幕分辨率后极大遍历了大量信息的同时显示。在现有的任务框架下,针对大屏的静态可视化设计是最佳选择。借助引人入胜的多分辨率技术(T3),其只需要一个视图(P1.3),较易使人理解。人们在远处即可获得对整体的认知,确定感兴趣区域后只需物理上稍微靠近屏幕即可获取更多的细节(P1.2),非常适合多人探索的任务(T1,T2)。


\paragraph {如何编码字号?}

词云中字的大小与对应的权重(如词频)具有明确的一对一关系,其作用是将权重具有显著差异的词目区分开来。由于词的长度不一,且词与词之间的相对位置不是对齐的,人们很难通过字号来准确地对比两个相似大小的词之权重~\supercite{Johann2009}。

存在多种权重与字号的对应关系$f$,满足单调递增且值域非负即可,如斜率为正的线性函数、排名函数、开平方函数等。我们认为,何种映射适于区分权重的层次是由数据自身决定的。例如,当存在一个特别大的权重异常值时,取开平方的即可缓和其与其他数据的差异,理论上优于一视同仁的线性函数。为了提高多分辨率词云对不同数据的适应能力,适应探索(T1)与呈现(T2,T3)的需求,我们认为不应局限权重与字号对应的形式,用户应能够自主选择合理的$f$。

根据Isenburg~\supercite{Isenberg2013}对WILD设备的实验结果(见图~\ref{fig:function}右侧),$64$像素是在$3$米开外区分远近文字的一个合理阈值。类似地,$f$的值域是相对显示屏幕尺寸及分辨率固定的,只需进一步确定$f$的函数类以及数据的定义域,即可构建映射。

\paragraph{文字方向有何约束?}早年的标签云多为水平行对齐,随着Wordle的诞生,文本任意旋转的词云获得了更为广泛的关注。然而,在我们大屏的设定下,由于在接近屏幕时,浏览较远处会产生一定的视角扭曲,为了使文字更易被识别(P2.3),我们限定词云为全水平布局。这也能简化基于搜索的词云布局算法,加快大屏上词云布局的计算。

\paragraph{下层词云如何布局?}

以ShapeWordle~\supercite{Wang2020}为代表的形变词云能够让第二级的文本紧凑布局于上级文本字形所天然形成的边界内部。但由于文字本身的空间覆盖率较低(P2.1),凑近时并不便阅读。且这种方法能够涵盖的子词云较少,视觉可扩展性低。而对图像拼接的方法来说,子文本需要在上层形状的限制下稍加变形,难以控制其字号与权重的对应关系,且不易读。

我们选择在父词云所确定的领域中放置子词云,以空间的相邻表示他们之间的关联(P1.1-2),指引自顶而下的探索(P1.2)。同时,我们还应保证子词云尽可能多地利用空白的位置(P2.1),并均匀排布(P2.2)。

\paragraph{如何使用颜色?}

尽管不同的灰度值足以实现双分辨率词云的基本要求,但我们仍选择使用颜色来为其增效。鲜艳多彩的颜色不仅引人注目(T2,T3),更能作为一个单独的视觉通道编码数据(T1,T2):色相可对应于离散型变量,而亮度或饱和度的变化在一定程度上也可对应于连续性变量。在一般的词云中,词的颜色仅用以区分各个短语。当考虑词义~\supercite{Barth2014,Hearst2019}或情感~\supercite{Kulahcioglu2019}等附加的属性时,色相有时会被用来表示一个类。在多分辨率词云中,类比于大多数词云的做法,我们使用不同的色相随机编码父词云。但对于与父词云具有关联的子词云,我们在父词云颜色的基础上添加扰动,用相似但稍有区分度的颜色来编码子词云,以此保持认知上的关联性(P1.1)。进一步地,我们还能通过调整词的亮度、色度等,区分开父词云与子词云,缓解他们对彼此的干扰(P2.3)。


\bigbreak
综上,我们基于双分辨率词云的具体任务与一般的设计准则对其生成时的一些关键问题进行了分析,作出了以下设计选择:
\begin{enumerate}[leftmargin=*]
	\renewcommand{\labelenumi}{\textbf{C\theenumi.}}
	\renewcommand{\labelenumii}{\textbf{ C\theenumi.\arabic{enumii}}}
	\item \textbf{考虑大屏上的静态可视化。}充分利用大屏的高分辨率特性,为多人浏览场景提供服务,为理解抽象文本尽多地提供线索。
	\item \textbf{保留权重信息,根据数据特点灵活选择映射类型。}为了将字的权重有效分层,双分辨率词云应在保留原始权重值的基础上给予用户调整权重-字号映射的空间。
	\item \textbf{水平布局。}防止大屏伴随的视角扭曲问题影响浏览,并提高算法效率。
	\item \textbf{子词云布局于父词云的邻域内。}在大屏上尽可能多地展示数据,同时保持父子词云在视觉上的关联性。
	\item \textbf{层次化赋色。}以差异较大的色相相区分父词云,在父词云颜色的基础上添加扰动分别子词云。其中,子词云色彩随其与父词云相对位置的改变有所调整。
\end{enumerate}
