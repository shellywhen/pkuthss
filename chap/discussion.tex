\chapter{讨论与展望}
\section{优势}
\begin{itemize}
	\item \textbf{展示层次文本数据}双分辨率词云是一种适用于具有双层关系的文本可视化方法,能够在现有词云概览的基础上进一步提供更多关联的信息,有利于观者更为准确地理解文本数据。
	\item \textbf{支持大规模数据}
	\item \textbf{具有艺术性}。经过基于LCH空间的色彩扰动,双分辨率词云
\end{itemize}


尽管在上文中没有提及,
\section{局限与挑战}
\begin{itemize}
	\item \textbf{纯静态展示,交互空间尚未探索。}
	\item \textbf{非实时加载,需要一定的计算时间,对大型显示设备所耗时更长。}
	\item \textbf{用户定制空间不足。}相较于成熟的词云生成软件,我们的创制系统目前对,我们通过一些参数
	\item \textbf{用户测试。}
\end{itemize}

\section{未来工作与潜在研究机会}
\begin{itemize}
	\item \textbf{探索双分辨率词云的交互空间。}
	\item \textbf{自动与交互之间。}利用Pops Out现象。
	\item \textbf{强化色彩编码。}在基于大屏的双分辨率词云的使用场景中,子
\end{itemize}
