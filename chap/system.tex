\chapter{双分辨率词云创制系统}
\label{sec:system}
双分辨率词云是一种任务驱动型的基础可视化,可应用于相当广泛的文本数据。尽管它可以在上一章算法的基础上被独立开发实现,但由于它是一种新颖的可视化方法,且所涉及的参数较多,新接触的人群难以理解参数与最终效果的关联,且尚无完备的优化准则来指导机器自动生成,因此不利于创作者与分析人员的了解与利用。为此,我们设计开发了一套基于用户-服务器模型的交互创制系统,~\toolname{},以支持用户对双分辨率词云的便捷创制。

\section{设计要求}
基于在设计双分辨率词云可视化的迭代过程中所积累的一些经验以及对现有的在线词云创建系统的参考,我们总结出了以下几项对双分辨率词云创制系统的设计要求,以方便用户快速生成双分辨率词云,并在一定程度上控制最终的效果。

\begin{enumerate}[leftmargin=*]
	\renewcommand{\labelenumi}{\textbf{D\theenumi.}}
	\renewcommand{\labelenumii}{\textbf{D\theenumi.\arabic{enumii}}}
	\item \textbf{提供灵活的数据处理功能}:由于双分辨率词云适用于探索式分析任务(T1)和数据分享任务(T2),因此创制系统也应相应地对数据格式有更灵活的支持,为具有不同需求的用户提供服务。
	\begin{enumerate}
		\item \textbf{纯文本语料库}---对毫无先验知识的纯文本,用户可直接上传,系统在简单处理后可视化其高频词及对应语句或文短以提供具有语境的概览,帮助用户找到数据中感兴趣的内容,挖掘数据背后的洞见。
		\item \textbf{格式化文本}---对数据已具备一定了解的用户亦可指定双层词云各自的具体内容,构建映射关系,展示相应的内容,促进交流。在这种情境下,系统应不再局限于简单的文本挖掘,支持用户自定义所需要可视化的内容。
	\end{enumerate}
	\item \textbf{支持对效果的调整}:在数据分享任务(T2)和寓教于乐任务(T3)中,可视化的最终效果需要符合设计者一定的心理预期,支持设计者对布局及格式的调整。
		\begin{enumerate}
		\item \textbf{参数推荐}---过于灵活的参数设置容易使可视化经验较少的用户茫然无措,而本研究所提出的双分辨率词云更是一种全新的方法,我们尚未充分地认识到参数与效果之间的对应关系。因此,我们需要在交互的过程中为用户提供更多的指引,基于固定的在不影响用户与系统交互的效率的前提下,尽可能地缩小用户需要探索的参数空间。
		\item \textbf{反馈效果}---及时反馈参数所对应的效果,避免让用户通过不断生成双分辨率词云的低效方式来调试。
		\item \textbf{自定义}---用户可具体指定各词语的可视化参数,如父文本词云(或与子文本词云)的布局位置或赋色方案。
	\end{enumerate}
\end{enumerate}

\section{系统概览}
\toolname{}是一个轻型的网站,由四个页面组成:主页、创制页面、画廊及说明页,采用前后端分离的客户-服务器模式。

\subsection{界面}
\subsection{交互逻辑}



\section{前后端交互}
\subsection{数据上传}
本节简要介绍该系统所支持的数据格式与布局参数接口。在下一章里,我们将通过几个具体的例子,说明该设定如何能够方便用户运用双分辨率词云可视化。
\paragraph	{\textbf{格式化文本}}
在前一章的数据模型设计下,\toolname{}以\texttt{json}作为双分辨率词云的的底层数据格式,并接收用户上传的符合规范的\texttt{json}数据(格式规范如下所示),以适应于更广泛的数据类型(D1.2)。这是一种独立于程序的数据交换语言,具有以下优点。
\begin{itemize}
	\item \textbf{易读性强}---它的层次结构清晰而简洁,易于理解。
	\item \textbf{普及性高}---\texttt{json}是ECMAScript®语言规范的标准之一,受几乎所有编程语言的支持,用户可在简单的数据预处理后生成相应文件。
	\item \textbf{适合网络上的数据交换}---在客户端与服务器端数据传输的过程中针对\texttt{json}型数据存在特定的压缩技术,能够节省带宽。
\end{itemize}

在此基础上,我们给予用户定制可视化的空间,允许他们指定父级词云的位置与颜色(D2.3)。

\begin{lstlisting}[language=json, caption=\texttt{json}上传数据格式规范,父级词云的位置与颜色域可为空]
{
	"keyword_list": {
		keyword_id: {
			"word": keyword_string,
			"weight": weight_of_word,
			"contexts": [
				context_id
			],
			"position": [x, y],
			"color": color_string
	},
	"context_list": [
		context_id: {
			"context": context_string,
			"weight": weight_of_context
		}
	]
}
\end{lstlisting}

\paragraph {\textbf{纯文本}}
在常见的词云创制网站,如Wordle~\footnote{\url{http://www.wordle.net}}和Word Art~\footnote{\url{https://wordart.com}}中,用户通过复制粘贴文本或指定文字来源网页超链接的方式来指定可视化的内容,限定使用\texttt{UTF-8}编码。注意到它们对于电子书、预印本等常见语料库的支持不足,我们还在系统内额外增添了\texttt{pdf}和\texttt{txt}格式的文件处理机制(D1.1),将他们转化为字符串。


\paragraph{\textbf{布局参数}}

\subsection{数据处理}
对于用户所上传的无任何先验知识的纯文本,系统默认将词频高的关键词作为父词云,其所在句子作为子词云。为了进一步确定用户所关心的细节程度,我们需要用户输入句子的分隔符,默认为句号、分号、感叹号与问号。

中文、日文、泰文等本文的一大特色在于,它们的句段中短语或词组不存在天然的分隔符。而由于语言的歧义性、词库有限等诸多难题,如何准确分词至今仍是热门的研究方向。为此,系统使用了性能较为良好的``结巴分词''\footnote{\url{https://github.com/fxsjy/jieba}}进行了分词操作以提取关键词。

\subsection{参数推荐}




\section{系统实现}
本研究所实现的双分辨率词云创制系统的\texttt{HTML5}网站在前端部分使用了轻量级的\texttt{Vue} 框架生成动态页面,以\texttt{JavaScript}语言协调用户与系统的交互,包括数据上传、参数输入、输入完整性检验与推荐,以及生成图像预览与下载。而后端部分则是基于\texttt{Flask}框架响应前端请求,对于自定义数据进行相应的处理,并根据参数返回生成的词云。后端所用程序语言为\texttt{Python},单独开发了双分辨率词云的类库,提供函数接口,在词云布局的迭代算法中使用了\texttt{Cython}以进行加速。

\draft{采访系统使用感受。}
